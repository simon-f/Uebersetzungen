\documentclass[paper=a4, fontsize=11pt]{scrartcl}

\usepackage[utf8x]{inputenc}
\usepackage[german]{babel}

\usepackage[osf,sc]{mathpazo}
\usepackage{courier}
\renewcommand{\sfdefault}{uop} % ---> URW Classico Optima Clone
\renewcommand{\rmdefault}{pplj} % ---> Mathpazo Palatino
\linespread{1.05}

\usepackage{amsmath,amsfonts,amsthm} % Math packages
\usepackage{booktabs} 
\usepackage{url} 

\usepackage{siunitx} 
\usepackage{titlesec} 
\usepackage{graphicx}
\usepackage{hyperref} 
\usepackage{changepage}
\usepackage{caption}
\usepackage{parallel}
%\setlength\parindent{0pt} 
\usepackage[modulo]{lineno}
\linenumbers

\begin{document}

\section*{Ein Besuch des Forums und des Marsfeldes\footnote{Lektion 7, Seite 42}}

\begin{Parallel}[v]{0.48\textwidth}{0.48\textwidth}
\ParallelLText{
Lūcius: ``Peregrīnus sum; ex parvō oppidō Italiae Rōmanam venī.
}
\ParallelRText{
Lucius: ``Ich bin fremd; ich bin aus einer kleinen Stadt Ialiens nach Rom gekommen.
} 
\ParallelPar
\ParallelLText{
Campum Mārtinum īgnōrō, etiam forum Rōmānum mihi īgnōtum est.
}
\ParallelRText{
Ich kenne das Marsfeld nicht, auch das Forum Romanum ist mir unbekannt.
}
\ParallelPar
\ParallelLText{
Forum vidēre cupiō, nam multa aedifica clāra in forō Rōmānō esse sciō.
}
\ParallelRText{
Ich möchte das Forum sehen, denn ich weiß, dass viele berühmte Gebäude auf dem Forum Romanum sind.
}
\ParallelPar
\ParallelLText{
Orō tē, Mārce, ī mēcum in forum!''
}
\ParallelRText{
Ich bitte dich, Marcus, gehe mit mir auf den Marktplatz!''
}
\ParallelPar
\ParallelLText{
Mārcus: ``Libenter tēcum eō.
}
\ParallelRText{
Markus: ``Gerne gehe ich mit dir.
}
\ParallelPar
\ParallelLText{
In forum īre tibique templa deōrum vel alia aedifica forī mōnstrāre mihi gaudiō est.''
}
\ParallelRText{
Es ist mir eine Freude, auf das Forum zu gehen und [dir] den Tempel der Götter oder sogar die anderen Gebäude des Forums zu zeigen.
}

\ParallelPar
\ParallelLText{
Mārcus cum Lūciō forum adit; viā arduā ad Capiōlium eunt; via amicīs magnō labōrī est.
}
\ParallelRText{
Marcus besucht mit Lucio das Forum; sie gehen die steile Straße zum Kapitol; der Weg macht den Freunden große Mühe.
}
\ParallelPar
\ParallelLText{
Dē Capitōliō forum spectant.
}
\ParallelRText{
Sie betrachten vom Kapitol herab auf das Forum.
}
\ParallelPar
\ParallelLText{
Lūcius: ``Vidēsne id magnum aedificum?
}
\ParallelRText{
Lucius: ``Siehst du dieses große Gebäu\-de?
}
\ParallelPar
\ParallelLText{
Dīc mihi nōmen aedificiī!''
}
\ParallelRText{
Sag mir den Namen des Gebäudes!''
}
\ParallelPar
\ParallelLText{
Mārcus: ``Nōmen aedificiī, 'Basilica Iūlia' est.
}
\ParallelRText{
Marcus: ``Der Name des Gebäudes ist 'Basilica Iūlia'.
}
\ParallelPar
\ParallelLText{
Magnum opus est.''
}
\ParallelRText{
Sie ist ein großes Werk.''
}
\ParallelPar
\ParallelLText{
Lūcius id opus multaque alia aedifica forī cum gaudiō spectat.
}
\ParallelRText{
Lucius betrachtet dieses Werk und viele andere Gebäude des Forums mit Freude.
}
\ParallelPar
\ParallelLText{
Tum amīcī forō exeunt, Campum Mār\-tium ineunt. 
}
\ParallelRText{
Dann verlassen die Freunde das Forum, sie gehen aufs Marsfeld.
}
\ParallelPar
\ParallelLText{
In Campō Mārtiō magnō theātrō appropinquant.
}
\ParallelRText{
Auf dem Marsfeld nähern sie sich einem großen Theater.
}
\ParallelPar
\ParallelLText{
Mārcus: ``Theātrum temporibus Pompēī aedificātum est.
}
\ParallelRText{
Marcus: ``Das Theater wurde zur Zeit Pompeis gebaut.
}
\ParallelPar
\ParallelLText{
Ecce, in tabulā nōmen Pompēī est.
}
\ParallelRText{
Sieh, auf der Tafel ist der Name Pompeis.
}
\ParallelPar
\ParallelLText{
Ita hominēs memoriam nominis Pompeī etiam hodiē servant.
}
\ParallelRText{
So bewahren die Menschen auch heute den Namen Pompeīs.  
}
\ParallelPar
\ParallelLText{
In theātrō opera et fābulae nōn sōlum poētārum antīquōrum, sed etiam hodiernōrum aguntur.''
}
\ParallelRText{
Im Theater werden nicht nur die Werke und Theaterstücke alter, sondern auch die heutiger Dichter aufgeführt.'' 
}
\ParallelPar
\ParallelLText{
Lūcius: ``Nōmina et opera poētārum clārōrum nōn īgnōrō.
}
\ParallelRText{
Lucius: ``Die Namen und Werke be\-rühm\-ter Dichter kenne ich genau. 
}
\ParallelPar
\ParallelLText{
Fābulae antīquōrum temporum mē dēlectant, nam memoria antīquōrum temporum mihi gaudiō est.''
}
\ParallelRText{
Die Theaterstücke alter Zeiten gefallen mir, denn die Erinnerung alter Zeiten sind mir eine Freude.''
}
\ParallelPar
\ParallelLText{
Marcus: ``Multa iam spectāvimus; cūncta vidēre hodiē nōbīs nōn licet, nam tempus nōbīs dēest.
}
\ParallelRText{
Marcus: ``Viel haben wir schon betrachtet; alles zu sehen ist heute nicht erlaubt, denn uns fehlt die Zeit.
}
\ParallelPar
\ParallelLText{
Itaque mēcum domum abī, amīce!''
}
\ParallelRText{
Daher gehe mit mir nach Hause fort, Freund!''
}
\end{Parallel}

\section*{Ein blutiges Volksvergnügen\footnote{Lektion 8, Seite 47f.}}
\begin{Parallel}[v]{0.48\textwidth}{0.48\textwidth}
\ParallelPar
\ParallelLText{
Tiberius, quī lūdōs gladiātōriōs valdē amat, cum Lūcio in amphitheātrum it.
}
\ParallelRText{
Tiberius der die Gladiatorenspiele sehr mag, geht mit Lucius ins Amphitheater.
}
\ParallelPar
\ParallelLText{
Nam hodiē imperātor lūdōs dat.
}
\ParallelRText{
Denn heute gibt der Kaiser Spiele.
}
\ParallelPar
\ParallelLText{
Tiberius Lūcium interrogat: ``Vidēsne bēstiās, quae ex Africa sunt?
}
\ParallelRText{
Tiberius fragt Lucius: ``Siehst du wilde Tiere, die aus Afrika sind?
}
\ParallelPar
\ParallelLText{
Spectā ursum, quōcum hodiē leō pūgnat.
}
\ParallelRText{
Betrachte die Bären, die heute mit dem Löwen kämpfen.
}
\ParallelPar
\ParallelLText{
Vidē! Gladiātōrēs veniunt!''
}
\ParallelRText{
Sieh! Die Gladiatoren kommen herein!''
}
\ParallelPar
\ParallelLText{
Spectātōrēs virōs, quī magnā et pulchrā pompā in arēnam intrant, clāmōre salūtant.
}
\ParallelRText{
Die Zuschauer begrüßen die Männer, die in einem großen und schönen Aufmarsch in die Arena eintreten, mit Geschrei.
}
\ParallelPar
\ParallelLText{
Tum imperātor sīgnum pūgnae dat. 
}
\ParallelRText{
Dann gibt der Kaiser das Zeichen des Kampfes.
}
\ParallelPar
\ParallelLText{
Duō gladiātōrēs, quibus mortifera arma sunt, prīmī in arēnā pūgnant: Thrāx et rētiārius.
}
\ParallelRText{
Zwei Gladiatoren, denen tödliche Waffen gehören, kämfen zuerst in der Arena: Ein Tranker und ein Netzkämpfer.
}
\ParallelPar
\ParallelLText{
Thrāx gladiō cum rētiārō pūgnat, cui rēte et fuscina arma sunt.
}
\ParallelRText{
Der Thranker kämpft mit Schwert gegen den Netzkämpfer, dem ein Netz und ein Dreizack gehören.
}
\end{Parallel}

\end{document}
