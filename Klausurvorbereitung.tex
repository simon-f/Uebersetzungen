\documentclass[paper=a4, fontsize=11pt]{scrartcl}

\usepackage[utf8x]{inputenc}
\usepackage[german]{babel}

\usepackage[osf,sc]{mathpazo}
\usepackage{courier}
\renewcommand{\sfdefault}{uop} % ---> URW Classico Optima Clone
\renewcommand{\rmdefault}{pplj} % ---> Mathpazo Palatino
\linespread{1.05}

\usepackage{amsmath,amsfonts,amsthm} % Math packages
\usepackage{booktabs} 
\usepackage{url} 

\usepackage{siunitx} 
\usepackage{titlesec} 
\usepackage{graphicx}
\usepackage{hyperref} 
\usepackage{changepage}
\usepackage{caption}
\usepackage{parallel}
%\setlength\parindent{0pt} 
\usepackage[modulo]{lineno}
\linenumbers

\begin{document}

\section*{Ein Besuch des Forums und des Marsfeldes\footnote{Lektion 7, Seite 42}}

\begin{Parallel}[v]{0.48\textwidth}{0.48\textwidth}
\ParallelLText{
Lūcius: ``Peregrīnus sum; ex parvō oppidō Italiae Rōmanam venī.
}
\ParallelRText{
Lucius: ``Ich bin fremd; ich bin aus einer kleinen Stadt Ialiens nach Rom gekommen.
} 
\ParallelPar
\ParallelLText{
Campum Mārtinum īgnōrō, etiam forum Rōmānum mihi īgnōtum est.
}
\ParallelRText{
Ich kenne das Marsfeld nicht, auch das Forum Romanum ist mir unbekannt.
}
\ParallelPar
\ParallelLText{
Forum vidēre cupiō, nam multa aedifica clāra in forō Rōmānō esse sciō.
}
\ParallelRText{
Ich möchte das Forum sehen, denn ich weiß, dass viele berühmte Gebäude auf dem Forum Romanum sind.
}
\ParallelPar
\ParallelLText{
Orō tē, Mārce, ī mēcum in forum!''
}
\ParallelRText{
Ich bitte dich, Marcus, gehe mit mir auf den Marktplatz!''
}
\ParallelPar
\ParallelLText{
Mārcus: ``Libenter tēcum eō.
}
\ParallelRText{
Markus: ``Gerne gehe ich mit dir.
}
\ParallelPar
\ParallelLText{
In forum īre tibique templa deōrum vel alia aedifica forī mōnstrāre mihi gaudiō est.''
}
\ParallelRText{
Es ist mir eine Freude, auf das Forum zu gehen und [dir] den Tempel der Götter oder sogar die anderen Gebäude des Forums zu zeigen.
}

\ParallelPar
\ParallelLText{
Mārcus cum Lūciō forum adit; viā arduā ad Capiōlium eunt; via amicīs magnō labōrī est.
}
\ParallelRText{
Marcus besucht mit Lucio das Forum; sie gehen die steile Straße zum Kapitol; der Weg macht den Freunden große Mühe.
}
\ParallelPar
\ParallelLText{
Dē Capitōliō forum spectant.
}
\ParallelRText{
Sie betrachten vom Kapitol herab auf das Forum.
}
\ParallelPar
\ParallelLText{
Lūcius: ``Vidēsne id magnum aedificum?
}
\ParallelRText{
Lucius: ``Siehst du dieses große Gebäu\-de?
}
\ParallelPar
\ParallelLText{
Dīc mihi nōmen aedificiī!''
}
\ParallelRText{
Sag mir den Namen des Gebäudes!''
}
\ParallelPar
\ParallelLText{
Mārcus: ``Nōmen aedificiī, 'Basilica Iūlia' est.
}
\ParallelRText{
Marcus: ``Der Name des Gebäudes ist 'Basilica Iūlia'.
}
\ParallelPar
\ParallelLText{
Magnum opus est.''
}
\ParallelRText{
Sie ist ein großes Werk.''
}
\ParallelPar
\ParallelLText{
Lūcius id opus multaque alia aedifica forī cum gaudiō spectat.
}
\ParallelRText{
Lucius betrachtet dieses Werk und viele andere Gebäude des Forums mit Freude.
}
\ParallelPar
\ParallelLText{
Tum amīcī forō exeunt, Campum Mār\-tium ineunt. 
}
\ParallelRText{
Dann verlassen die Freunde das Forum, sie gehen aufs Marsfeld.
}
\ParallelPar
\ParallelLText{
In Campō Mārtiō magnō theātrō appropinquant.
}
\ParallelRText{
Auf dem Marsfeld nähern sie sich einem großen Theater.
}
\ParallelPar
\ParallelLText{
Mārcus: ``Theātrum temporibus Pompēī aedificātum est.
}
\ParallelRText{
Marcus: ``Das Theater wurde zur Zeit Pompeis gebaut.
}
\ParallelPar
\ParallelLText{
Ecce, in tabulā nōmen Pompēī est.
}
\ParallelRText{
Sieh, auf der Tafel ist der Name Pompeis.
}
\ParallelPar
\ParallelLText{
Ita hominēs memoriam nominis Pompeī etiam hodiē servant.
}
\ParallelRText{
So bewahren die Menschen auch heute den Namen Pompeīs.  
}
\ParallelPar
\ParallelLText{
In theātrō opera et fābulae nōn sōlum poētārum antīquōrum, sed etiam hodiernōrum aguntur.''
}
\ParallelRText{
Im Theater werden nicht nur die Werke und Theaterstücke alter, sondern auch die heutiger Dichter aufgeführt.'' 
}
\ParallelPar
\ParallelLText{
Lūcius: ``Nōmina et opera poētārum clārōrum nōn īgnōrō.
}
\ParallelRText{
Lucius: ``Die Namen und Werke be\-rühm\-ter Dichter kenne ich genau. 
}
\ParallelPar
\ParallelLText{
Fābulae antīquōrum temporum mē dēlectant, nam memoria antīquōrum temporum mihi gaudiō est.''
}
\ParallelRText{
Die Theaterstücke alter Zeiten gefallen mir, denn die Erinnerung alter Zeiten sind mir eine Freude.''
}
\ParallelPar
\ParallelLText{
Marcus: ``Multa iam spectāvimus; cūnc\-ta vidēre hodiē nōbīs nōn licet, nam tempus nōbīs dēest.
}
\ParallelRText{
Marcus: ``Viel haben wir schon betrachtet; alles zu sehen ist heute nicht erlaubt, denn uns fehlt die Zeit.
}
\ParallelPar
\ParallelLText{
Itaque mēcum domum abī, amīce!''
}
\ParallelRText{
Daher gehe mit mir nach Hause fort, Freund!''
}
\end{Parallel}

\vspace{2em}
\section*{Ein blutiges Volksvergnügen\footnote{Lektion 8, Seite 47f.}}
\begin{Parallel}[v]{0.48\textwidth}{0.48\textwidth}
\ParallelPar
\ParallelLText{
Tiberius, quī lūdōs gladiātōriōs valdē amat, cum Lūcio in amphitheātrum it.
}
\ParallelRText{
Tiberius, der die Gladiatorenspiele sehr mag, geht mit Lucius ins Amphitheater.
}
\ParallelPar
\ParallelLText{
Nam hodiē imperātor lūdōs dat.
}
\ParallelRText{
Denn heute gibt der Kaiser Spiele.
}
\ParallelPar
\ParallelLText{
Tiberius Lūcium interrogat: ``Vidēsne bēstiās, quae ex Africa sunt?
}
\ParallelRText{
Tiberius fragt Lucius: ``Siehst du wilde Tiere, die aus Afrika sind?
}
\ParallelPar
\ParallelLText{
Spectā ursum, quōcum hodiē leō pūg\-nat.
}
\ParallelRText{
Betrachte die Bären, die heute mit dem Löwen kämpfen.
}
\ParallelPar
\ParallelLText{
Vidē! Gladiātōrēs veniunt!''
}
\ParallelRText{
Sieh! Die Gladiatoren kommen herein!''
}
\ParallelPar
\ParallelLText{
Spectātōrēs virōs, quī magnā et pulchrā pompā in arēnam intrant, clāmōre salū\-tant.
}
\ParallelRText{
Die Zuschauer begrüßen die Männer, die in einem großen und schönen Aufmarsch in die Arena eintreten, mit Geschrei.
}
\ParallelPar
\ParallelLText{
Tum imperātor sīgnum pūgnae dat. 
}
\ParallelRText{
Dann gibt der Kaiser das Zeichen des Kampfes.
}
\ParallelPar
\ParallelLText{
Duō gladiātōrēs, quibus mortifera arma sunt, prīmī in arēnā pūgnant: Thrāx et rētiārius.
}
\ParallelRText{
Zwei Gladiatoren, denen tödliche Waffen gehören, kämfen zuerst in der Arena: Ein Tranker und ein Netzkämpfer.
}
\ParallelPar
\ParallelLText{
Thrāx gladiō cum rētiārō pūgnat, cui rēte et fuscina arma sunt.
}
\ParallelRText{
Der Thranker kämpft mit Schwert gegen den Netzkämpfer, dem ein Netz und ein Dreizack gehören.
}
\ParallelPar
\ParallelLText{
Spectātōrēs, quōrum numerus magnus est, virōs magnō clāmōre ad pūgnam incitant.
}
\ParallelRText{
Die Zuschauer, deren Zahl groß ist, treiben die Männer mit großem Geschrei zum Kampf an.
}
\ParallelPar
\ParallelLText{
Tandem rētiārius Thrācem, cuius gladius frāctus est, rēte involvit.
}
\ParallelRText{
Endlich wickelt der Netzkämpfer den Thranker, dessen Schwert zerbrochen ist, in sein Netz ein.
}
\ParallelPar
\ParallelLText{
Vir miser victōris clēmentiam implōrat.
}
\ParallelRText{
Der arme Mann erfleht die Gnade des Siegers.
}
\ParallelPar
\ParallelLText{
Spectātōrēs imperātorem virum miserum mittere iubent, nam magnā virtūte pūgnāvit.
}
\ParallelRText{
Die Zuschauer befehlen dem Kaiser, den armen Mann freizulassen, denn er hat mir großer Tapferkeit gekämpft.
}
\ParallelPar
\ParallelLText{
Tum aliī gladiātōrēs cum bēstiās pūg\-nant. 
}
\ParallelRText{
Nun kämpfen die anderen Gladiatoren mit den Tieren.
}
\ParallelPar
\ParallelLText{
Virī bēstiās, quae ē portīs carceris provolant, sagittīs caedunt.
}
\ParallelRText{
Die Männer töten die Tiere, die aus der Käfigstür hervorstürmen, mit Pfeilen.
}
\ParallelPar
\ParallelLText{
Tandem leō et ursus in arēnam currunt. 
}
\ParallelRText{
Endlich läuft der Löwe und der Bär in die Arena.
}
\ParallelPar
\ParallelLText{
Bēstiae, quās duō servī ad pūgnam incitant, diū pūgnant. 
}
\ParallelRText{
Die Tiere, die zwei Slaven zum Kampf antreiben, kämpfen lange.  
}
\ParallelPar
\ParallelLText{
Leō vincit; servī ursum, quī multīs vulneribus cōnfectus est, ex arēnā trahunt.
}
\ParallelRText{
Der Löwe siegt; Slaven tragen den Bären, der durch viele Verletzungen geschwächt ist, aus der Arena.
}
\ParallelPar
\ParallelLText{
Tiberius Lūcium interrogat: ``Dēlectant\-ne tē lūdī, amīce?''
}
\ParallelRText{
Tiberius fragt Lucius: ``Erfreuen dich die Spiele, mein Freund?''
}
\ParallelPar
\ParallelLText{
Respondet Lūcius: ``Minimē dēlectant. 
}
\ParallelRText{
Lucius antwortet: ``Sie erfreuen [mich] wenig.  
}
\ParallelPar
\ParallelLText{
Egō spectācula, quae vidēmus, nōn pulchra, sed inhūmāna esse putō.
}
\ParallelRText{
Ich halte das Schauspiel, das wir uns ansehen, nicht für schön, sondern für unmenschlich.
}
\ParallelPar
\ParallelLText{
Glōria, quam illī virī miserī petunt, glōria mala est.
}
\ParallelRText{
Der Ruhm, den jenen armen Männer anstreben; dieser Ruhm ist schlecht.
}
\ParallelPar
\ParallelLText{
Egō quidem dīcō: Quī homō amphitheātrum init, bēstia ex amphitheātrō exit.''
}
\ParallelRText{
Ich, ich sage allerdings: ``Wer das Amphitheater als Mensch betritt, verlässt das Amphitheater als Tier.''
}
\end{Parallel}

\vspace{2em}
\section*{Eine Schreckensnachricht aus Germanien\footnote{Lektion 9, Seite 53}}
\begin{Parallel}[v]{0.48\textwidth}{0.48\textwidth}
\ParallelPar
\ParallelLText{
Lūcius: ``Nōnne audīvistī, Mārce, nūn\-tium malum, quem mercātōrēs ē Germā\-niā apportavērunt?''
}
\ParallelRText{
Lucius: ``Hast du denn die schlechte Nachricht nicht gehört, Marcus, die Kaufleute aus Germanien mitgebracht haben?''
}
\ParallelPar
\ParallelLText{
Mārcus: ``Audīvī, sed rēs certās nōn cognōvī.
}
\ParallelRText{
Mārcus: ``Ich hörte [sie], aber ich erfuhr keine sicheren Sachen.
}
\ParallelPar
\ParallelLText{
Iam rūmor nōbīs magnō terrōrī fuit.''
}
\ParallelRText{
Schon das Gerücht brachte uns großen Schrecken.
}
\ParallelPar
\ParallelLText{
Lūcius: ``Ineptē dīcis, Gāī. 
}
\ParallelRText{
Lucius: ``Du redest Unsinn Gaius.
}
\ParallelPar
\ParallelLText{
Augustus fīnēs imperiī multīs legiōni\-bus bene dēfendit.
}
\ParallelRText{
Augustus verteidigte die Grenzen des Reiches mit vielen Legionen gut.
}
\ParallelPar
\ParallelLText{
Egō mīles sub Tiberiō Caesare in Ger\-māniā fuī.
}
\ParallelRText{
Ich war Soldat unter Tiberius, Caesar in Germanien.
}
\ParallelPar
\ParallelLText{
Castella multa ad Rhēnum posuimus.''
}
\ParallelRText{
Wir haben viele Festungen am Rhein errichtet.''
}
\ParallelPar
\ParallelLText{
Gāius: ``Pūgnāvistīne cum Germānīs?
}
\ParallelRText{
Gaius: ``Hast du mit den Germanen gekämpft? 
}
\ParallelPar
\ParallelLText{
 Pūgnīsne interfuistī?''  
}
\ParallelRText{
Hast du an Schlachten teilgenommen?''  
}
\ParallelPar
\ParallelLText{
Lūcius: ``Interfuī. Germānōs multīs pūgnīs superāvimus.''
}
\ParallelRText{
Ich habe teilgenommen. Wir besiegten die Germanen in vielen Schlachten.
}
\ParallelPar
\ParallelLText{
Gāius: ``Tē mīlitem bonum fuisse nōn īgnōrō, Lūcī; sed vōs interrogō: Nōnne spectāvistis gladiātōrēs Germanānōs, quī nūper in arēnā pūgnāvērunt?''
}
\ParallelRText{
Gaius: ``Dass du ein guter Soldat warst, weiß ich Lucius; aber ich frage euch: Habt ihr nicht die Germanischen Gladiatoren betrachtet, die neulich in der Arena kämpften.
}
\ParallelPar
\ParallelLText{
Mārcus: ``Ita. Servī Germānī, quōs spectāvimus, magnā virtūte pūgnāvērunt.
}
\ParallelRText{
Marcus: ``Ja. Die Germanischen Sklaven, die wir betrachteten, kämpften mit großer Tapferkeit.
}
\ParallelPar
\ParallelLText{
Itaque metuō virtūtem Germānōrum.''
}
\ParallelRText{
Deswegen fürchte ich die Tapferkeit der Germanen.''
}
\end{Parallel}

\vspace{2em}
\section*{Ein Überlebender der Varusschlacht berichtet\footnote{Lektion 9, Seite 53f}}
\begin{Parallel}[v]{0.48\textwidth}{0.48\textwidth}
\ParallelPar
\ParallelLText{
  Multīs diēbus post mīles, quī ē clāde Vārianā fugā sē servāvit, nārrat:
}
\ParallelRText{
  Viele Tage später erzählt ein Soldat, der sich aus der Varusschlacht durch Flucht rettete:
}
\ParallelPar
\ParallelLText{
  ``Arminius, dux Cheruscōrum et amī\-cus populī Rōmānī, Vārō imperātorī nūn\-tiā\-vit paucās gentēs Germānās contrā populum Rōmānum coniūrāvisse.
}
\ParallelRText{
  ``Arminius, Anführer der Cherusker und Freund des römischen Volkes, meldete dem Feldherren Varus, dass sich wenige germanische Stämme gegen das römische Volk verschworen haben. 
}
\ParallelPar
\ParallelLText{
  Vārus statim cum legiōnibus castrīs exiit et ad gentēs īnfēstās contendit.
}
\ParallelRText{
  Varus ging sofort mit den Legionen aus den Lagern und eilte zu den feindlichen Stämmen.
}
\ParallelPar
\ParallelLText{
  Arminius nōbīs iter mōnstrāvit.
}
\ParallelRText{
  Arminius zeigte uns den Weg.
}
\ParallelPar
\ParallelLText{
  Magnō labōre per silvās dēnsās iimus, castra in palūdibus posuimus.
}
\ParallelRText{
  Mit großer Mühe gingen wir durch die dichten Wälder, in den Sümpfen errichteten wir ein Lager.
}
\ParallelPar
\ParallelLText{
  Multōs mīlitēs Rōmānōs silvās, imbrēs, palūdēs magis quam Germānōs metuisse putō.
}
\ParallelRText{
   Ich glaube, dass viele römischen Soldaten, die Wälder, die Regenfälle und die Sümpfe mehr als die Germanen fürchteten.
 }
\ParallelPar
\ParallelLText{
  Subitō Germānī īnfēstās armīs ē silvīs dēnsīs provolāvērunt.
}
\ParallelRText{
  Plötzlich stürmten die Gemanen mit [gezückten] Waffen aus den dichten Wäl\-dern hervor.
}
\ParallelPar
\ParallelLText{
  Sērō Varus dux malam Arminiī fidem cognōvit.
}
\ParallelRText{
  Zu spät erkannte Varus die schlechte Treue Arminus.
}
\ParallelPar
\ParallelLText{
  Mīlitēs ducēsque sē fortiter dēfendē\-runt, sed paucī ē clāde Vāriānā superfuērunt et ad Rhēnum rediērunt.''
}
\ParallelRText{
  Die Soldaten und der Anführer verteidigten sich tapfer, aber wenige überlebten die Varusschlacht und kehrten zum Rhein zurück.
}
\end{Parallel}

\vspace{2em}
\section*{Das Ende des Romulus\footnote{Lektion 10, Seite 59}}
\begin{Parallel}[v]{0.48\textwidth}{0.48\textwidth}
\ParallelPar
\ParallelLText{
  Antīquīs temporibus rēgēs cīvitātem Rōmānam regēbant.
}
\ParallelRText{
  In alten Zeiten regierten Könige die römische Bevölkerung.
}
\ParallelPar
\ParallelLText{
  Rōmulus, conditor urbis, Rōmae et prīmus Rōmanōrum rēx, urbem novam et lībertātem cīvum ab hostibus semper dēfendēbat imperiumque populī Rōmānī augēbat.
}
\ParallelRText{
  Romulus, Gründer der Stadt Roms und erster Königs der Römer, verteidigte die neue Stadt und die bürgerlichen Freiheiten immer von Feinden und baute das Imperium des Volkes Roms auf.
}
\ParallelPar
\ParallelLText{
  Quem Rōmānī semper magnō in honōre habēbant.
}
\ParallelRText{
  Ihn hielten die Römer immer in großer Ehre.
}
\ParallelPar
\ParallelLText{
  Aliquandō rēx cōpiās Rōmānās recēn\-sē\-re cupīvit et cīvēs Rōmānōs in Campō Martiō vocāvit. 
}
\ParallelRText{
  Einmal wollte der König die Truppen Roms mustern und rief die Bürger Roms auf das Marsfeld.
}
\ParallelPar
\ParallelLText{
  Multās hōrās in tribūnālī sedēbat, ē quō cōpiās recēnsēbat.
}
\ParallelRText{
  Er saß viele Stunden im Feldherrensitz, von wo aus er die Truppen musterte.
}
\ParallelPar
\ParallelLText{
  Subitō magna tempestās appropinquāvit, nimbus dēnsus rēgem occultāvit.
}
\ParallelRText{
  Plötzlich näherte sich ein großes Gewitter und eine dichte Wolke verbarg den König.
}
\ParallelPar
\ParallelLText{
  Deinde Rōmulus in terrīs nōn iam fuit.
}
\ParallelRText{
  Dann war Romulus schon nicht [mehr] auf der Erde.
}
\ParallelPar
\ParallelLText{
  Diū cīvēs Rōmānī in Campō Mārtiō stābant et tacēbant.
}
\ParallelRText{
  Lange saßen die Bürger Roms auf dem Marsfeld und schwiegen.
}
\ParallelPar
\ParallelLText{
  Tandem mīlitēs senātōrēsque magnō cum timōre domum iērunt.
}
\ParallelRText{
  Endlich gingen die Soldaten und Senatoren mit großer Furcht nach Hause.
}
\ParallelPar
\ParallelLText{
  In itinere alius alium iterum iterumque interrogābat:
}
\ParallelRText{
  Auf dem Weg fragte der eine den anderen immer wieder:
}
\ParallelPar
\ParallelLText{
  ``Nōnne etiam tū in Campō Mārtiō aderās?''
}
\ParallelRText{
  ``Hast du auch das Marsfeld besucht?''
}
\ParallelPar
\ParallelLText{
  ``Aderam; tōtum diem prope tribūnal stābam.''
}
\ParallelRText{
  ``Ich habe es besucht; den ganzen Tag stand ich in der Nähe des Feldherrensitzes.''
}
\ParallelPar
\ParallelLText{
  ``Quid dīcis?
}
\ParallelRText{
  ``Was sagst du?
}
\ParallelPar
\ParallelLText{
  Num deī Rōmulum, fīlium Mārtis deī et ducem nostrum clārum, ē terrā sustulērunt?''
}
\ParallelRText{
  Haben etwa die Götter Romulus, den Sohn des Gottes Mars und unseren berühmten Führer, von der Erde entrückt?''
}
\ParallelPar
\ParallelLText{
  ``Egō quidem patrēs Rōmulum necāvis\-se putō.
}
\ParallelRText{
  ``Ich, ich glaube, dass die Stadtväter Romulus getötet haben.
}
\ParallelPar
\ParallelLText{
  Nōnne rēgem nostrum patribus invidiae esse saepe audīebāmus?''
}
\ParallelRText{
  Hörten wir nicht oft, dass unser König den Neid der Stadtväter auf sich zieht?''  
}
\end{Parallel}

\vspace{2em}
\section*{Eine Botschaft aus dem Jenseits\footnote{Lektion 10, Seite 59}}
\begin{Parallel}[v]{0.48\textwidth}{0.48\textwidth}
\ParallelPar
\ParallelLText{
  Paucīs diēbus post Proculus Iūlius senātor in cōntiōne nārrāvit:
}
\ParallelRText{
  Wenige Tage später erzählte der Senator Proculus Julius in der Volksversammlung:
}
\ParallelPar
\ParallelLText{
  ``Prīmā hōrā diēī per Campum Mārtinum ībam et dē Rōmulō, rēge nostrō, cum dolōre cōgitābam.
}
\ParallelRText{
  ``In der ersten Stunde des Tages ging ich durch das Marsfeld und dachte mit Schmerz über unseren König Romulus nach''
}
\ParallelPar
\ParallelLText{
  Quī subitō mihi appāruit mēque vocāvit:
}
\ParallelRText{
  Er erschien mir plötzlich und sagte mir:
}
\ParallelPar
\ParallelLText{
  'Nūntiā Rōmānīs: Deī Rōmam meam caput orbis terrārum esse et cūnctīs populīs lēgēs dare volunt.'
}
\ParallelRText{
  'Melde den Römern: Die Götter wollen, dass mein Rom Hauptstadt des Erdkreises ist und allen Völkern Gesetze gibt.' 
}
\ParallelPar
\ParallelLText{
  Diū stābam, metuēbam. 
}
\ParallelRText{
  Lange stand ich, fürchtete ich.
}
\ParallelPar
\ParallelLText{
  Tum Rōmulus iterum sublīmis abiit.''
}
\ParallelRText{
  Dann ging Romulus plötzlich wieder in die Höhe fort.''
}
\ParallelPar
\ParallelLText{
  Quō ex tempore Rōmānī memoriam Rōmulī, patris patriae, semper sacram habēbant.
}
\ParallelRText{
  Seit dieser Zeit halten die Römer die Erinnerung an Romulus, dem Vater der Heimat, immer heilig.
}
\end{Parallel}

\vspace{2em}
\section*{Von der Königsherrschaft zur Republik\footnote{Lektion 10, Seite 62f}}
\begin{Parallel}[v]{0.48\textwidth}{0.48\textwidth}
\ParallelPar
\ParallelLText{
  Post Romulum, conditorem urbis Romae et patrem patriae, alii reges rem publicam regebant.
}
\ParallelRText{
  Nach Romulus, dem Gründer der Stadt Roms und Vater der Heimat, regierten andere Könige den Staat.
}
\ParallelPar
\ParallelLText{
  Numa Pompilius cultum deorum instituit multaque templa in urbe aedificavit.
}
\ParallelRText{
  Numa Pompilius ordnete den Kult der Götter baute und viele Tempel in der Stadt.
}
\ParallelPar
\ParallelLText{
  Ancus Marcius etiam Latinos, qui vicini populi Romani erant, regebat.
}
\ParallelRText{
  Ancus Marcius regierte sogar die Latiner, die die Nachbarn des römischen Volkes waren.
}
\ParallelPar
\ParallelLText{
  Sed alii populi urbem Romam virtutemque Romanorum cum invidia spectabant.
}
\ParallelRText{
  Aber die anderen Völker betrachteten die Stadt Rom und die Tapferkeit der Römer mit Neid.
}
\ParallelPar
\ParallelLText{
  Itaque amici et hostes Romanis non deerant.
}
\ParallelRText{
  Deswegen fehlten den Römern weder Freunde noch Feinde.
}
\ParallelPar
\ParallelLText{
  Servium Tullium regem murum primum circa Romam aedificavisse Romani putabant.
}
\ParallelRText{
  Die Römer glaubten, dass der König Servius Tullius eine erste Mauer um Rom gebaut hatte.
}
\ParallelPar
\ParallelLText{
  Populus Tarquinium Superbum, regem ultimum, quem propter superbiam timebat, ex urbe pepulit et ita rem publicam liberavit.
}
\ParallelRText{
  Das Volk vertrieb Tarquinius Superbus, den letzten König, den es wegen seines Übermuts fürchtete, aus der Stadt und befreite die Republik.
}
\ParallelPar
\ParallelLText{
  Libertatem novam Romani diu servabant et defendebant.
}
\ParallelRText{
  Die Römer bewahrten und verteidigten die neue Freiheit lange.
}
\end{Parallel}

\vspace{2em}
\section*{Eine fahrlässige Körperverletzung mit Todesfolge \footnote{Lektion 11, Seite 64}}
\begin{Parallel}[v]{0.48\textwidth}{0.48\textwidth}
\ParallelPar
\ParallelLText{
  Et iuvnes et viri Romani saepe cum gaudio pila ludebant.
}
\ParallelRText{
  Sowohl die jungen Männer, als auch die Männer Roms spielten oft mit Freude Ball.
}
\ParallelPar
\ParallelLText{
  Etiam in viis locisque publicis interdum pilas iactabant, quamquam ibi laborabant fabri et erat magna copia hominum.
}
\ParallelRText{
  
}
\end{Parallel}

\end{document}