\documentclass[paper=a4, fontsize=11pt]{scrartcl}

\usepackage[utf8x]{inputenc}
\usepackage[german]{babel}

\usepackage[osf,sc]{mathpazo}
\usepackage{courier}
\renewcommand{\sfdefault}{uop} % ---> URW Classico Optima Clone
\renewcommand{\rmdefault}{pplj} % ---> Mathpazo Palatino
\linespread{1.05}

\usepackage{amsmath,amsfonts,amsthm} % Math packages
\usepackage{booktabs} 
\usepackage{url} 

\usepackage{siunitx} 
\usepackage{titlesec} 
\usepackage{graphicx}
\usepackage{hyperref} 
\usepackage{changepage}
\usepackage{caption}
\usepackage{parallel}
%\setlength\parindent{0pt} 
\usepackage[modulo]{lineno}
\linenumbers

\begin{document}
\begin{Parallel}[v]{0.48\textwidth}{0.48\textwidth}
\ParallelLText{
  \textbf{Dē bellō Varianō} \\
}
\ParallelRText{
  \textbf{Vom Krieg des Varus}
}
\ParallelPar
\ParallelLText{
  Germānī iterum iterumque novam patriam in Italiā quaerēbant; quō ex tempore Rōmānīs semper perīculō erant.
}
\ParallelRText{
  Die Germanen suchten immer wieder neues Vaterland in Italien; seit dieser Zeit waren [schwebten] die Römer immer in Gefahr.
}
\ParallelPar
\ParallelLText{
  Itaque annō nōnō p. Chr. n. Augustus Quīntīlium Varum iubet cum tribus legionibus contrā Germānōs iter facere.
}
\ParallelRText{
  Deshalb befiehlt Augustus Quintilius Varus im Jahre 9 n.Chr. sich mit drei Legionen gegen die Germanen auf den Weg zu machen.
}
\ParallelPar
\ParallelLText{
  In numerō mīlitum erat M. Caelius. }
\ParallelRText{
  Unter vielen Soldaten war M.Caelius.
}
\ParallelPar
\ParallelLText{
  Quamquam iam multis pūgnīs interfuerat, cum Germānīs nōndum pūgnāverat.
}
\ParallelRText{
  Obwohl er schon an vielen Schlachten teilgenommen hatte, hatte er noch nicht mit den Germanen gekämpft.  
}
\ParallelPar
\ParallelLText{
  Itaque nōn sine aliquō timōre iter fēcit.
}
\ParallelRText{
  Deswegen machte er sich nicht gänzlich ohne Angst auf den Weg.
}
\ParallelPar
\ParallelLText{
  Cōnstat legiōnem M. Caeliī Arminio dūce fīnēs Cheruscōrum petivisse; quōs hostēs inhūmānōs Rōmānīs fuisse scīmus.
}
\ParallelRText{
  Es ist bekannt, dass die Legionen des M. Caelius unter Führung von Arminius zu den Grenzen der Cherusker hineilten; wir wissen, dass diese unmenschlichen Feinde der Römer waren. 
}
\ParallelPar
\ParallelLText{
  Sērō Vārus dux malam Arminiī fidem cognōvit, qui pūgnā trium diērum trēs legiōnēs dēlēvit sīgna-que dōmum dē duxit.
}
\ParallelRText{
  Zu spät erkannte der Führer Varus die schlechte Treue des Arminus, dieser zerstörte im dreitägigen Kampf drei Legionen und führte das Feldzeichen nach Hause.
}
\ParallelPar
\ParallelLText{
  Postquam M. Caelius bellō Variānō cecidit, frāter monumentum fēcit quō memoriam mortis frātris servāvit.
}
\ParallelRText{
  Nachdem M. Caelius im Varianischen Krieg gefallen war, fertigte der Bruder ein Grabmal an, damit das Gedenken an den toten Bruder überlebte.
}
\end{Parallel}
\end{document}
 