\documentclass[paper=a4, fontsize=11pt]{scrartcl}

\usepackage[utf8x]{inputenc}
\usepackage[german]{babel}

\usepackage[osf,sc]{mathpazo}
\usepackage{courier}
\renewcommand{\sfdefault}{uop} % ---> URW Classico Optima Clone
\renewcommand{\rmdefault}{pplj} % ---> Mathpazo Palatino
\linespread{1.05}

\usepackage{amsmath,amsfonts,amsthm} % Math packages
\usepackage{booktabs} 
\usepackage{url} 

\usepackage{siunitx} 
\usepackage{titlesec} 
\usepackage{graphicx}
\usepackage{hyperref} 
\usepackage{changepage}
\usepackage{caption}
\usepackage{parallel}
%\setlength\parindent{0pt} 
\usepackage[modulo]{lineno}
\linenumbers

\begin{document}

\begin{Parallel}[v]{0.48\textwidth}{0.48\textwidth}
\ParallelLText{\Large Dē bellō Varianō}
\ParallelRText{\Large Übersetzung}
\ParallelPar
\vspace{2em}

\ParallelLText{
  Germānī iterum iterumque novam patriam in Italiā quaerēbant; quō ex tempore Rōmānīs semper perīculō erant.
}
\ParallelRText{
  Die Germanen suchten immer wieder das neue Vaterland in Italien auf; wohin sogleich die Römer immer in Gefahr schwebten.
}
\ParallelPar
\ParallelLText{
  Itaque annō nōnō p. Chr. Augustus Quīntīlium Varum iubet cum tribus legionibus contrā Germānōs iter facere.
}
\ParallelRText{
  Deshalb befiehlt Augustus Quintilius Varus im Jahre 9 v.Chr. mit drei Legionen gegen die Germanen auf den Weg zu machen.
}
\ParallelPar
\ParallelLText{
  In numerō mīlitum erat M. Caelius, quamquam iam multis pūgnīs interfuerat, cum Germānīs nōndum pūgnāverat, itaque nōn sine qliquō timōre iter fēcit.
}
\ParallelRText{
  Als Soldat war M. Caelius, obwohl er schon an vielen Schlachten teilgenommen hatte, noch nicht mit den Germanen kämpfte.
}






\end{Parallel}
\end{document}
 